\hypertarget{anwesend}{%
\section{Anwesend}\label{anwesend}}

Peter, Paul, Clara

\hypertarget{top1}{%
\section{Top1}\label{top1}}

\Blindtext[1][3]

\hypertarget{safafs}{%
\subsection{SAFAFS}\label{safafs}}

\Blindtext[2][7]

\hypertarget{subsub}{%
\subsubsection{subsub}\label{subsub}}

\Blindtext[2][5]

\hypertarget{top2}{%
\section{Top2}\label{top2}}

\begin{enumerate}
\def\labelenumi{\arabic{enumi}.}
\item
  \textbf{bla} \Blindtext[1][5]
\item
  bli
\item
  blub
\item
  blo
\item
  bu
\end{enumerate}

\hypertarget{top3}{%
\section{Top3}\label{top3}}

\begin{itemize}
\item
  bla
\item
  bli
\item
  blob
\end{itemize}

\hypertarget{an-exhibit-of-markdown}{%
\section{An exhibit of Markdown}\label{an-exhibit-of-markdown}}

This note demonstrates some of what
\href{http://daringfireball.net/projects/markdown/}{Markdown} is capable
of doing.

\emph{Note: Feel free to play with this page. Unlike regular notes, this
doesn't automatically save itself.}

\hypertarget{basic-formatting}{%
\subsection{Basic formatting}\label{basic-formatting}}

Paragraphs can be written like so. A paragraph is the basic block of
Markdown. A paragraph is what text will turn into when there is no
reason it should become anything else.

Paragraphs must be separated by a blank line. Basic formatting of
\emph{italics} and \textbf{bold} is supported. This \emph{can be
\textbf{nested} like} so.

\hypertarget{lists}{%
\subsection{Lists}\label{lists}}

\hypertarget{ordered-list}{%
\subsubsection{Ordered list}\label{ordered-list}}

\begin{enumerate}
\def\labelenumi{\arabic{enumi}.}
\tightlist
\item
  Item 1
\item
  A second item
\item
  Number 3
\item
  Ⅳ
\end{enumerate}

\emph{Note: the fourth item uses the Unicode character for
\href{http://www.fileformat.info/info/unicode/char/2163/index.htm}{Roman
numeral four}.}

\hypertarget{unordered-list}{%
\subsubsection{Unordered list}\label{unordered-list}}

\begin{itemize}
\tightlist
\item
  An item
\item
  Another item
\item
  Yet another item
\item
  And there's more\ldots{}
\end{itemize}

\hypertarget{paragraph-modifiers}{%
\subsection{Paragraph modifiers}\label{paragraph-modifiers}}

\hypertarget{code-block}{%
\subsubsection{Code block}\label{code-block}}

\begin{verbatim}
Code blocks are very useful for developers and other people who look at code or other things that are written in plain text. As you can see, it uses a fixed-width font.
\end{verbatim}

You can also make \texttt{inline\ code} to add code into other things.

\hypertarget{quote}{%
\subsubsection{Quote}\label{quote}}

\begin{quote}
Here is a quote. What this is should be self explanatory. Quotes are
automatically indented when they are used.
\end{quote}

\hypertarget{headings}{%
\subsection{Headings}\label{headings}}

There are six levels of headings. They correspond with the six levels of
HTML headings. You've probably noticed them already in the page. Each
level down uses one more hash character.

\hypertarget{headings-can-also-contain-formatting}{%
\subsubsection{\texorpdfstring{Headings \emph{can} also contain
\textbf{formatting}}{Headings can also contain formatting}}\label{headings-can-also-contain-formatting}}

\hypertarget{they-can-even-contain-inline-code}{%
\subsubsection{\texorpdfstring{They can even contain
\texttt{inline\ code}}{They can even contain inline code}}\label{they-can-even-contain-inline-code}}

Of course, demonstrating what headings look like messes up the structure
of the page.

I don't recommend using more than three or four levels of headings here,
because, when you're smallest heading isn't too small, and you're
largest heading isn't too big, and you want each size up to look
noticeably larger and more important, there there are only so many sizes
that you can use.

\hypertarget{urls}{%
\subsection{URLs}\label{urls}}

URLs can be made in a handful of ways:

\begin{itemize}
\tightlist
\item
  A named link to \href{http://www.markitdown.net/}{MarkItDown}. The
  easiest way to do these is to select what you want to make a link and
  hit \texttt{Ctrl+L}.
\item
  Another named link to \href{http://www.markitdown.net/}{MarkItDown}
\item
  Sometimes you just want a URL like \url{http://www.markitdown.net/}.
\end{itemize}

\hypertarget{horizontal-rule}{%
\subsection{Horizontal rule}\label{horizontal-rule}}

A horizontal rule is a line that goes across the middle of the page.

\begin{center}\rule{0.5\linewidth}{\linethickness}\end{center}

It's sometimes handy for breaking things up.

\hypertarget{images}{%
\subsection{Images}\label{images}}

Markdown can also contain images. I'll need to add something here
sometime.

\hypertarget{finally}{%
\subsection{Finally}\label{finally}}

There's actually a lot more to Markdown than this. See the official
\href{http://daringfireball.net/projects/markdown/basics}{introduction}
and \href{http://daringfireball.net/projects/markdown/syntax}{syntax}
for more information. However, be aware that this is not using the
official implementation, and this might work subtly differently in some
of the little things.
